% Cualquier duda, contactarse con matias.milia@estudiante-flacso.mx
\documentclass[a4paper,12pt]{report}
\usepackage[spanish]{babel} 
\usepackage[T1]{fontenc} % agrego este paquete porque, de lo que pude leer en los foros, este es para la salida de texto y no tener problemas
\usepackage[utf8]{inputenc} % tuve que cambiar de Latin1 a utf8 porque, al parecer, el texto estaba codificado en esta norma y el archivo no compilaba. Al parecer, la diferencia entre UTF8 y Latin1 es el largo de los caracteres porque en utf8 ocupan más bytes que en Latin1. 
\usepackage{adjustbox} % permite ajustar las tablas a la página o proporciones de la misma
\usepackage[hidelinks]{hyperref} %ESCONDE URLs al poner los links
\usepackage{vmargin}
\usepackage{colortbl} % TABLAS CON COLOR
\usepackage{arydshln} % TABLAS
\usepackage{array, multirow, makecell, hhline}
\usepackage{ltablex} % TableX para la proporción en multirow
\usepackage{ragged2e} % IDEM
\usepackage{geometry} % IDEM
\usepackage[inline]{enumitem} % IDEM
\usepackage[version=3]{mhchem} % este paquete es para poder escribir los compuestos químicos 
\usepackage{dblfloatfix} % para que flote la tabladesafío
\usepackage{multirow} % lo pide https://www.latex-tables.com/#
\usepackage{colortbl} % color para las casillas de las tablas de https://www.latex-tables.com/#
\usepackage{rotating} %rota  https://www.latex-tables.com/#
\usepackage{lipsum} % para que se pueda hacer un salto de página
\usepackage{pdfpages} % este paquete permite insertar páginas PDF en el archivo
\usepackage{enumitem} % sirve para utilizar números romanos, en las enumeraciones "\begin{enumerate}[label=\roman*.]" también hay otros
\usepackage{apacite}
\usepackage{graphicx} % gráficosa
\usepackage{subfigure} % subfiguras
\usepackage{epstopdf} % Insertar PDF como imagenes
\usepackage{booktabs}
\usepackage{float} % para hacer 'flotar' las tablas
\usepackage[flushleft]{threeparttable} % para las notas en las tablas
\usepackage{footnote}
\usepackage{lscape} % para poder hacer que la página vaya horizontal
\usepackage{xcolor} % para modificar color
\usepackage{longtable} % permite tablas de más de una página
\usepackage{titletoc}% %%%%% TITLETOC es para poner "Capítulo" adelante del número de capítulo
\titlecontents{chapter}% <section-type>
  [0pt]% <left>
  {\bfseries}% <above-code>
  {\chaptername\ \thecontentslabel:\quad}% <numbered-entry-format>
  {}% <numberless-entry-format>
  {\hfill\contentspage}% <filler-page-format>
	\usepackage{lipsum}
	\usepackage{setspace}
  %%%%
\usepackage[sort&compress]{natbib}
\usepackage{savesym} % sirve para agregar los símbolos ? y ?
\savesymbol{approx} % sirve para agregar los símbolos ? y ?
\usepackage{MnSymbol} % sirve para agregar los símbolos ? y ?
\usepackage{makeidx} % Para el índice analítico 
%\usepackage{fancyhdr} % Encabezados y pies de página. Más info acá: http://minisconlatex.blogspot.com/2013/01/como-editar-los-encabezados-y-pies-de.html y acá: https://ondahostil.wordpress.com/2017/05/22/lo-que-he-aprendido-encabezados-y-pies-de-pagina-en-latex/
\restoresymbol{approx}{amssymb}
\renewcommand{\thefootnote}{\arabic{footnote}} % 1, 2, 3... para las notas al pie
\setpapersize{A4}
\setmargins{2.5cm}       % margen izquierdo
{1.5cm}                        % margen superior
{16.5cm}                      % anchura del texto
{23.42cm}                    % altura del texto
{10pt}                           % altura de los encabezados
{1cm}                           % espacio entre el texto y los encabezados
{0pt}                             % altura del pie de página
{2cm}                           % espacio entre el texto y el pie de página
\renewcommand{\baselinestretch}{1,5} %interlineado
\usepackage{pgfplots} % BOXPLOT Este es para hacer 'gráficos de caja y bigote', 'boxplots' o 'box and whisker chart' 
\usepgfplotslibrary{colormaps} % COLORES BOXPLOT
\pgfplotsset{compat=1.9} % BOXPLOT
\usepgfplotslibrary{statistics} % BOXPLOT
\usetikzlibrary{pgfplots.colormaps}
\pgfplotsset{compat=1.14}  % BOXPLOT WIdTH
\usetikzlibrary{backgrounds}  % BOXPLOT WIDTH
\usepackage{blindtext}  % BOXPLOT WIDTH


%%%%%%%%%%%%%%%%%%%%%%%%%%%%%%%%%%%%%%%%%%%%%%%%%%%%%%%%%%%%%%%%%%%%%%%%%%%%%%%%
%%%% 						TITULO 																		%%%%%%%%%%%
%%%%%%%%%%%%%%%%%%%%%%%%%%%%%%%%%%%%%%%%%%%%%%%%%%%%%%%%%%%%%%%%%%%%%%%%%%%%%%%%

\makeatletter
\def\namedlabel#1#2{\begingroup
   \def\@currentlabel{#2}%
   \label{#1}\endgroup
}
\makeatother
\begin{document}
\pagenumbering{gobble} % para que no ponga número de página 

\begin{titlepage}
   \begin{center}
       \vspace*{0.2cm}
        \includegraphics[width=0.2\textwidth]{LogoFlacso.png} \\ % LOGO DE FLACSO ACTUALIZADO A 2019
       \vspace*{0.2cm}
	\textsc{Facultad Latinoamericana de Ciencias Sociales} \\
	\textsc{sede académica México} \\
	Maestría en... \\. %% PROGRAMA ACADÉMICO
	?? Promoción\\ % PROMOCIÓN o Cohorte
	2019 -- 2020 \\ % AÑOS
       \vspace{0.3cm}	
       \textbf{\large{Titulo.}  \\ \normalsize \textsl{Subtítulo.} \\}
        \vspace{0.3cm}
        Tesis para obtener el grado de Doctor en Ciencias Sociales\\
        \vspace{0.3cm}
        Presenta: \\
        \textbf{Autor}\\
        \vspace{0.2cm}
	\normalsize {Directores de tesis:} \\
	\normalsize {\textsf{Dra. 1}},
	\normalsize {\textsf{Dr. Dr. 2}}.
	\\
	\normalsize {Lectores de tesis:} \\
	\normalsize {\textsf{Dr. 1},}
	\normalsize {\textsf{Dra. 2}.}
       \vspace{0.4cm}\\
Seminario de Investigación: \\ % Completar nombre del seminario 
Línea de Investigación:  \\ % completar nombre de la línea
       \vspace{0.5cm}
Ciudad de México,  \\ % Fecha
       \vspace{0.5cm}
\footnotesize{Este Doctorado fue realizado gracias a una beca otorgada por \\
el Consejo Nacional de Ciencia y Tecnología (\textsc{conacyt})}\\
       \vfill
 
   \end{center}
\end{titlepage}

%%%%%%%%%%%%%%%%%%%%%%%%%%%%%%%%%%%%%%%%%%%%%%%%%%%%%%%%%%%%%%%%%%%%%%%%%%%%%%%%
%%%% 						ÍNDICES 																		%%%%%%%%%%%
%%%%%%%%%%%%%%%%%%%%%%%%%%%%%%%%%%%%%%%%%%%%%%%%%%%%%%%%%%%%%%%%%%%%%%%%%%%%%%%%

\tableofcontents
\pagebreak
\cleardoublepage
\listoffigures % indice de figuras

\cleardoublepage
\listoftables

\cleardoublepage

%%%%%%%%%%%%%%%%%%%%%%%%%%%%%%%%%%%%%%%%%%%%%%%%%%%%%%%%%%%%%%%%%%%%%%%%%%%%%%%%
%%%% 						DEDICATORIA																	%%%%%%%%%%%
%%%%%%%%%%%%%%%%%%%%%%%%%%%%%%%%%%%%%%%%%%%%%%%%%%%%%%%%%%%%%%%%%%%%%%%%%%%%%%%%

\thispagestyle{plain}
\begin{flushright}
       \vspace*{3.5cm}
\large
\textsl{A los chinos, por la tinta.}
       \vspace*{0.4cm}   
       \end{flushright}

%%%%%%%%%%%%%%%%%%%%%%%%%%%%%%%%%%%%%%%%%%%%%%%%%%%%%%%%%%%%%%%%%%%%%%%%%%%%%%%%
%%%% 						AGRADECIMIENTOS 																		%%%%%%%%%%%
%%%%%%%%%%%%%%%%%%%%%%%%%%%%%%%%%%%%%%%%%%%%%%%%%%%%%%%%%%%%%%%%%%%%%%%%%%%%%%%%

\chapter*{Agradecimientos} % si no queremos que añada la palabra "Capitulo"
\phantomsection
\addcontentsline{toc}{chapter}{Agradecimientos}

A mis directores, ... 

A mis lectores, ... 

A mi seminario y sus coordinadores, ... 

A mis entrevistados...

A los profesores que apoyaron... 


    \textsl{De corazón, ¡muchas, muchas gracias!}


\cleardoublepage

%%%%%%%%%%%%%%%%%%%%%%%%%%%%%%%%%%%%%%%%%%%%%%%%%%%%%%%%%%%%%%%%%%%%%%%%%%%%%%%%
%%%% 						ABSTRACT ESPAÑOL															%%%%%%%%%%%
%%%%%%%%%%%%%%%%%%%%%%%%%%%%%%%%%%%%%%%%%%%%%%%%%%%%%%%%%%%%%%%%%%%%%%%%%%%%%%%%

\thispagestyle{plain}
\begin{center}
    \Large
    \textbf{TITULO.}\\
    \vspace{0.4cm}
    \large
SUBTITULO  \\
 \vspace{0.8cm}

    \textbf{Resumen} \\
 \normalsize
 \vspace{0.3cm}
 \begin{justify}

RESUMEN
 %Máximo 200 palabras
 \end{justify}
 \vspace{0.6cm}
   
\textsl{Palabras clave:} \textsc{ }\\
    %Máximo 10 palabras clave
    \cleardoublepage
	

%%%%%%%%%%%%%%%%%%%%%%%%%%%%%%%%%%%%%%%%%%%%%%%%%%%%%%%%%%%%%%%%%%%%%%%%%%%%%%%%
%%%% 						ABSTRACT EN INGLÉS															%%%%%%%%%%%
%%%%%%%%%%%%%%%%%%%%%%%%%%%%%%%%%%%%%%%%%%%%%%%%%%%%%%%%%%%%%%%%%%%%%%%%%%%%%%%%

\large
 \vspace{0.8cm}
    \textbf{Abstract} \\
     \normalsize
 \vspace{0.3cm}
  \begin{justify}
  
  
  \end{justify}
    \textsl{Keywords:}    \textsc{}

\end{center}


%%%%%%%%%%%%%%%%%%%%%%%%%%%%%%%%%%%%%%%%%%%%%%%%%%%%%%%%%%%%%%%%%%%%%%%%%%%%%%%%%%%%%%%%%%%%%%%%%%%%%%%%
%%%%%%%%%%%%%%%%%%%%%%%%%%%% COMIENZA LA TESIS %%%%%%%%%%%%%%%%%%%%%%%%%%%%%%%%%%%%%%%%%%%%%%%%%%%%%%%%%%%%%%%%%%%%%%%%%%%%%%%%%%%%%%%%%%%%%%%%%%%%%%%%%%%%%%%%%%%%%%%%%%%%%%%%%%%%%%%%%%%%%%%%%%%%%%%%%%%%%%%%%%%%%%%%%%%%%%%%%%%%

\chapter*{Introducción. \\{\huge subtítulo introducción.}}
% La introducción no sale numerada y tiene un subtítulo después de \huge 
\phantomsection
\addcontentsline{toc}{chapter}{Introducción.} % Entre estas llaves va lo que quiero que salga en el índice 
\pagenumbering{roman} %%%% Esta línea está haciendo que todas las páginas estén numeradas con números romanos

\section*{Sección 1 Introducción } 
\phantomsection
\addcontentsline{toc}{section}{Sección 1 Introducción} % Entre estas llaves va lo que quiero que salga en el índice 
. 

\section*{Sección 2 Introducción}
\phantomsection
\addcontentsline{toc}{section}{Sección 2 Introducción} % Entre estas llaves va lo que quiero que salga en el índice 


\clearpage % salto de página
\pagenumbering{arabic} % cambia a la numeración a ARÁBICA 


%%%%%%%%%%%%%%%%%%%%%%%%%%%%%%%%%%%%%%%%%%%%%%%%%%%%%%%%%%%%%%%%%%%%%%%%%%%%%%%%
%%%% 						Capítulo 1															%%%%%%%%%%%
%%%%%%%%%%%%%%%%%%%%%%%%%%%%%%%%%%%%%%%%%%%%%%%%%%%%%%%%%%%%%%%%%%%%%%%%%%%%%%%%

\chapter{Marco Teórico.}\label{Cap1} % Label permite hacer un enlace en el texto al capítulo usando \ref{Cap1}

%%%%%%%%%%%%%%%%%%%%%%%%%%%%%%%%%%%%%%%%%%%%%%%%%%%%%%%%%%%%%%%%%%%%%%%%%%%%%%%%
%%%% 						Capítulo 2															%%%%%%%%%%%
%%%%%%%%%%%%%%%%%%%%%%%%%%%%%%%%%%%%%%%%%%%%%%%%%%%%%%%%%%%%%%%%%%%%%%%%%%%%%%%%

\chapter{Resultados.}\label{Cap2}% Label permite hacer un enlace en el texto al capítulo usando \ref{Cap2}


\cleardoublepage

\chapter*{Conclusiones} % este es el título que aparece antes de que se den las citas bibliográficas
\phantomsection
\addcontentsline{toc}{chapter}{Conclusiones} % este es el título que aparece en el índice


\cleardoublepage
\phantomsection
\addcontentsline{toc}{chapter}{Bibliografía} % este es el título que aparece en el índice
\renewcommand{\bibname}{Bibliografía} % este es el título que aparece antes de que se den las citas bibliográficas
\bibliographystyle{apacite} % Tipo de citación que deseo que el programa haga
\bibliography{DCS_Tesis} % Aquí va el archivo .bib donde está toda la bibliotrafía
\end{document}